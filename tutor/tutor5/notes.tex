\documentclass{article}
\usepackage[T1]{fontenc}
\usepackage{hyperref}
\usepackage{microtype}
%\usepackage[margin=1.5in]{geometry}

% libertine
\usepackage{libertine}
\usepackage[libertine]{newtxmath}

% begin sections in a new page
\usepackage{titlesec}
\newcommand{\sectionbreak}{\clearpage}

% customize list margins
\usepackage{enumitem}
\setlist[itemize]{align=parleft,left=0pt..1.5em}

\hypersetup{
  pdftitle={Notes for Critical Thinking},
  pdfauthor={Samuel Tam},
  pdfsubject={CGE14411 Critical Thinking},
  pdfkeywords={Arguments, ambiguity, distortions, fallacies, issues, vagueness, validity}
}

\begin{document}

\section*{Critical Thinking}

\section{Lecture 1}

\subsection{What is Critical Thinking?}
\textbf{Critical thinking} consists of an awareness of a set of interrelated critical questions, plus the ability and willingness to ask and answer them at appropriate times. Critical thinking refers to the following:
\begin{itemize}
  \item Awareness of a set of interrelated critical questions;
  \item Ability to ask and answer these critical questions in an appropriate manner;
  \item Desire to actively use the critical questions.
\end{itemize}

\subsection{Thinking Styles}
\textbf{The sponge approach} emphasizes knowledge \emph{acquisition}. \textbf{The panning-for-gold approach} stresses \emph{active interaction} with knowledge as it is being acquired. The two approaches complement each other.

\subsection{Weak-Sense and Strong-Sense Critical Thinking}
\textbf{Weak-sense critical thinking} is the use of critical thinking to \emph{defend} your current beliefs.
\textbf{Strong-sense critical thinking} is the use of the same skills to \emph{evaluate} all claims and beliefs, especially your own.

\subsection{Values and People}
\textbf{Values} are unstated ideas that people see as worthwhile. They provide \emph{standards of conduct} by which we measure the \emph{quality} of human behavior. People are drawn to objects, experiences, and actions because of ideas we value, and we expect others to meet the same. \textbf{Managed reasoning} means that the reasoning is being selected so as to reach a particular conclusion.

\subsection{Stereotypes}
By \textbf{stereotyping}, we allege that because a person is a member of a particular group, they must have a \emph{specific set of characteristics}. Bias toward any issue or controversy is immediately formed when these people are involved, because stereotypes load the issue in advance, \emph{prior} to the reasoning.

\subsection{Halo Effect}
\textbf{The halo effect} refers to our tendency to recognize one positive or negative quality or trait of a person, and then associate that quality or trait with \emph{everything} about that person.

\subsection{Belief Perseverance}
\textbf{Belief perseverance} is the tendency to stick to or persevere personal beliefs, and that we are biased from the start of an exchange in favor of our \emph{current} opinions and conclusions. Part of it comes from exaggerated sense of competence.

Exaggerated loyalty to own beliefs is one of the sources of \textbf{confirmation bias}, our tendency to see only that evidence that confirms what we already believe as being good evidence, which combined leads to \textsc{weak-sense critical thinking}.

\subsection{Availability Heuristic}
\textbf{The availability heuristic} refers to the mental shortcut we use again and again of forming conclusions based on whatever information is immediately available to us.

\textbf{The recency effect} is closely related, which is that the immediately available basis for our thinking is often the \emph{most recent} piece of information we have encountered.

\subsection{Answering the Wrong Question}
Failing to communicate effectively by providing an \emph{immediate automatic} answer that comes easily to mind, but fail to respond to the question that was asked. Because it was unconsciously substituted by \emph{our} question.

\subsection{Egocentrism}
\textbf{Egocentrism} refers to the central role we assign to our world, as opposed to the experiences and opinions of others. When we make and evaluate arguments, we often forget our audience as we focus on what we know and what we know how to do.

\textbf{The curse of knowledge} is that we cannot recall what it is like when we did not know what we now know, and it particularly cripples the commiuncation between critical thinkers and those who have not learned critical thinking.

\subsection{Wishful Thinking}
Branded as the biggest obstacle to critical thinking, \textbf{wishful thinking} is the tendency to form unrealistic beliefs to match concepts we wish to be true. Denial patterns, anxieties and fears serve as a barrier keeping us from the truth.

One variant is \textbf{magical thinking}. People tend to use \emph{magic} for causal explanations, or try to influence things that science cannot. It happens when people feel most powerless to understand or alter a situation, so they blindly believe in somebody or some new idea to make change.

\section{Lecture 2}

\subsection{Issue}
An \textbf{issue} is a \emph{question} or \emph{controversy} responsible for the conversation or discussion. It is the stimulus for what is being said.

\subsection{Descriptive Issue}
\textbf{Descriptive issues} are those that raise questions about the \emph{accuracy of descriptions} of the past, present, or future. The reasonings should be general facts, general laws and evidence supporting the premises.

\subsection{Prescriptive Issue}
\textbf{Prescriptive issues} are those that raise questions about what we \emph{should do} or what is \emph{right or wrong}, \emph{good or bad}. The reasonings are based on values and principles of \emph{ethical} and \emph{moral} judgement.

\subsection{Conclusion}
A \textbf{conclusion} is the message that the speaker or writer wishes you to accept. You cannot determine the worth of a conclusion until you identify the reasons. \textbf{Thesis statement} is a usual place for conclusion. \textbf{Assertions} are claims unsupported by any reason.

\subsection{Reasons}
\textbf{Reasons} are explanations or rationales for why we should believe a particular conclusion. If  reasons are an \emph{afterthought} following the selection of a conclusion, it is \textsc{weak-sense critical thinking}.

\subsection{Argument}
An \textbf{argument} consists of a \textbf{conclusion} and the \textbf{reasons} that allegedly supporting it. Arguments have several characteristics:
\begin{itemize}
  \item They have intent, for us to believe and call for a reaction.
  \item Their quality varies, which critical thinking evaluates.
  \item They have two essential visible components---a \textsc{conclusion} and \textsc{reasons}.
\end{itemize}

\section{Lecture 3}

\subsection{Ambiguity}
\textbf{Ambiguity} refers to the existence of multiple possible meanings for a word or phrase. The more \emph{abstract} a word or phrase, the more likely it is to be susceptible to multiple interpretations. 
We often misunderstand what we read or hear because we presume that the meaning of words is obvious.

\subsection{Equivocation}
If alternative meanings can be \emph{substituted} into the \emph{reasoning} structure, and changing the meaning makes a difference in how well a reason supports the conclusion, then an important ambiguity is located. \textbf{Equivocation} occurs when a key term changes meaning in the middle of an argument.

\subsection{Other types of Ambiguity}
\begin{itemize}
  \item \textbf{Lexical ambiguities} are cases where a single word or name has more than one meaning in a language.
  \item \textbf{Referential ambiguity} arises when the context does not make it clear what a pronoun or quantifier is referring to.
  \item \textbf{Syntactic ambiguity} occurs when there is more than one way to interpret the grammatical structure of an expression. 
\end{itemize}

\subsection{Vagueness}
A term is \textbf{vague} if it has an \emph{imprecise boundary}. A term can be vague even though it is not ambiguous, such as the Atlantic Ocean. Think of ambiguity as possibility, and vagueness as blurriness.

\subsection{Incomplete Meaning}
Terms with \textbf{incomplete meanings} presuppose certain standards of comparison, but leave the standards \emph{unspecified}.

\subsection{Conceptual Distortion}
\textbf{Distortion} is a matter of misconstruing the meaning of words, such as giving an incorrect reportive definition.

\subsection{Loaded Language}
Terms and phrases have both \emph{denotative} and \emph{connotative} meanings, which refers to the agreed-upon explicit descriptive referents for use of the word, and the emotional associations that we have to a term or phrase, respectively. \textbf{Loaded terms} trigger strong \emph{emotional reactions} and outweigh their descriptive meanings.

\subsection{Other types of Distortion}
\begin{itemize}
  \item \textbf{Weasel words} are cases where the ordinary meaning of a word is changed inappropriately in the middle of a discussion, usually in response to some counterexample or an objection.
  \item \textbf{Quoting out of context} misrepresents what other people have said.
  \item \textbf{Category mistake} is to assign a property to an object when it is logically impossible for an object of that kind to have the property in question.
  \item \textbf{Reincation} is a variant of category mistake, describes treating an abstract idea or property as if it were a concrete physical object.
\end{itemize}

\subsection{Empty Meaning}
\textbf{Empty meaning} is a case in which words are used without serving any useful purpose or providing little information. \emph{Empty question} occurs when the question serve no useful purpose. \emph{Empty statements} are trivially true, thus provide no concrete information.

\textbf{Analytic statements} are empty statements, but are important to logic and mathematics, and useful for language learning. They can also convey useful \textbf{conversational implicatures}, like expressing uncertainty or to emphasize the available options.

\subsection{Gobbledygook}
The word \textbf{gobbledygook} was coined to describe obscure and convoluted language full of jargon. The problem comes when buzzwords are used to obscure, hide, or inflate ideas and actually hinder accurate and effective communication.

\section{Lecture 4}

\subsection{Necessary Condition}
To say that $p$ is a \textbf{necessary} condition for $q$ is to say that the occurrence of $p$ is required for the occurrence of $q$. In other words, \emph{$q$ requires $p$}; \textsc{or} \emph{$p$, if $q$}. However $p$ can be true even if $q$ is not true. We write:
$$ p \Leftarrow q $$
\textsc{e.g.} Having four sides is necessary for being a square.

\subsection{Sufficient Condition}
If $p$ is a \textbf{sufficient} condition for $q$, this means the occurrence of $p$ guarantees the occurrence of $q$. In other words, \emph{if $p$ then $q$}. However $q$ is true does not guarantee that $p$ is true. We write:
$$ p \Rightarrow q $$
\textsc{e.g.} Being a square is sufficient for having four sides.

\subsection{The Write-off Fallacy}
\textbf{The write-off fallacy} is to argue that something is not important, because it is not necessary or not sufficient for something else that is good or valuable.

\subsection{Deductive Arguments}
From general information, to arrive at a more specific conclusion. Logical necessity is absolute and not a matter of degree, thus reasoning provides \textbf{definite conclusion}.

\subsection{Inductive Arguments}
Bases on specific incidents, makes a broader generalization that is considered probable; though the conclusion may not be absolutely, and only holds with a certain degree of probability. Provides \textbf{general patterns} instead of definite conclusion.

\section{Lecture 5}

\subsection{Validity}
An argument is valid if and only if there is no logically possible situation in which the premises are true and the conclusion is false.

\subsection{Soundness}
An argument is sound if and only if it is valid and \emph{all the premises are true}.

\section{Lecture 6}

\subsection{Patterns of Vaild Arguments}

\subsubsection{Modus Ponens}
If $p$ implies $q$, and $p$ is true, then $p$ is true. Also known as \textbf{implication elimination} or \textbf{affirming the antecedent}.
$$ \frac{p \Rightarrow q, p}{\therefore q} $$

\subsubsection{Modus Tollens}
If $p$ implies $q$, and $q$ is false, then $p$ is false. Also known as \textbf{denying the consequent}.
$$ \frac{p \Rightarrow q, \neg q}{\therefore \neg p} $$

\subsubsection{Disjunctive Syllogism}
Either $p$ or $q$ is true, or both; if $p$ is false, then $q$ is true.
\begin{align*}
\frac{p \lor q, \neg p}{\therefore q} &&
\frac{p \lor q, \neg q}{\therefore p}
\end{align*}

\subsubsection{Hypothetical Syllogism}
If $p$ implies $q$ and $q$ implies $r$, then $p$ implies $r$.
$$ \frac{p \Rightarrow q, q \Rightarrow r}{\therefore p \Rightarrow r} $$

\subsubsection{Constructive Dilemma}
If $p$ implies $q$ and $r$ implies $s$; but either $p$ or $r$ is true; then either $q$ or $s$ is true.
$$ \frac{p \Rightarrow q, r \Rightarrow s, p \lor r}{\therefore q \lor s} $$

\subsubsection{Destructive Dilemma}
If $p$ implies $q$ and $r$ implies $s$; but either $p$ or $r$ is false; then either $q$ or $s$ is false.
$$ \frac{p \Rightarrow q, r \Rightarrow s, \neg q \lor \neg s}{\therefore \neg p \lor \neg r} $$

\subsubsection{Bidirectional Dilemma}
If $p$ implies $q$ and $r$ implies $s$; but either $p$ is true or $s$ is false; then either $q$ is true or $r$ is false.
$$ \frac{p \Rightarrow q, r \Rightarrow s, p \lor \neg s}{\therefore q \lor \neg r} $$

\subsubsection{Proof by Contradiction}
Also known as \emph{reductio ad absurdum}. First assume that $S$ is true, then show that the assumption itself leads to a contradiction, or a claim that is false or absurd. Finally, conclude that $S$ must be false.

\subsection{Patterns of Invaild Arguments}

\subsubsection{Affirming the Consequent}
Not to be confused with \textsc{modus ponens} or \textsc{affirming the antecedent}.
$$ \frac{p \Rightarrow q, q}{\therefore p} $$

\subsubsection{Denying the Antecedent}
Not to be confused with \textsc{modus tollens} or \textsc{denying the consequent}.
$$ \frac{p \Rightarrow q, \neg p}{\therefore \neg q} $$

\subsection{Generalizations}
\begin{itemize}
  \item \textbf{Universal}: Every $F$ is $G$; all $F$s are $G$s.
  \item \textbf{Existential}: Some $F$ is $G$; at least one $F$ is $G$. 
  \item \textbf{Statistial}: A certain proportion of $F$s are $G$s.
\end{itemize}

\subsubsection{Vaild Patterns}
\begin{align*}
\frac{F \subset G, x \in F}{\therefore x \in G} &&
\frac{F \subset G, x \not\in G}{\therefore x \not\in F} &&
\frac{F \subset G, G \subset H}{\therefore F \subset H}
\end{align*}

\subsubsection{Invaild Patterns}
\begin{align*}
\frac{F \subset G, x \in G}{\therefore x \in F} &&
\frac{F \subset G, x \not\in F}{\therefore x \not\in G} &&
\frac{\exists x \subset G \text{ for some } x \in F, \exists x \subset H \text{ for some } x \in G}{\therefore \exists x \subset H \text{ for some } x \in F}
\end{align*}

\section{Lecture 7}

\section{Lecture 8}

\end{document}
